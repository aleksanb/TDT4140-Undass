\section{TDT4145 Datamodellering og databasesystemer}

\subsection{Databasedelen av fellesprosjektet}

\subsubsection{Konseptuel datamodell Innlevering: Se it's learning}

Dere skal lage en konseptuel databasemodell som oppfyller alle punkter i kravspesifikasjonen som er gitt i fellesprosjektheftet. Det er to ting som skal leveres: 

\begin{enumerate}

\item
Et diagram som viser datamodellen. Det er valgfritt om dere vil bruke UML eller EER, og om dere vil tegne for hånd eller bruke modelleringsverktøy. Ta med alle entitetsklasser, relasjonsklasser, attributter, kardinaliteter, eventuelle eksistensavhengigheter og svake entitetsklasser.

\item
Et dokument som beskriver hvordan modellen oppfyller kravspesifikasjonen. For hvert nummererte krav i kravspesifikasjonen skal det kort (et par linjer burde holde) forklares hvordan modellen deres oppfyller kravet.  

\end{enumerate}

For å bli godkjent må modellen deres oppfylle absolutt alle krav i kravspesifikasjonen. Kravoppfyllelsesdokumentet er en forutsetning for at vi skal kunne sjekke dette noenlunde enkelt, så innleveringer uten dette vil ikke bli godkjent. 

Besvarelsen skal leveres samlet i PDF. Husk å ta med gruppenummer og navn på alle i gruppa på innleveringen.

\subsubsection{DB2: Logisk databaseskjema Innlevering: Se it's learning}

Den konseptuelle modellen fra Oppgave 1 skal omformes til et logisk databaseskjema, i form av et SQL-skript som skal kunne brukes til å generere en relasjonsdatabase i MySQL. Det er to ting som skal leveres: 

\begin{enumerate}

\item
Et kjørbart SQL-skript som genererer databasen. Husk alle primær- og fremmednøkler, og eventuelt andre restriksjoner (constraints). 

\item
Oppgave 1-innleveringen. Hvis dere har endret modellen siden oppgave 1, lag en revidert versjon av modellen. Dere står fritt til å endre datamodellen til enhver tid, men alle krav må alltid være oppfylt. Husk å oppdatere kravoppfyllelsedokumentet, og marker endringene slik at vi slipper å sjekke alt på nytt. Vitsen med å levere inn oppgave 1 igjen, er at vi skal kunne kontrollere at dere har laget SQL-koden korrekt i henhold til modellen. Det må derfor være samsvar mellom modellen og SQL-koden for at innleveringen skal bli godkjent. 

\end{enumerate}

SQL-scriptet skal lastes opp som en separat tekstfil og resten skal samles i PDF.  Husk å skrive gruppenummer og navn på alle i gruppa på innleveringen.