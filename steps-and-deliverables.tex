\section{Trinn og leveranser}

Hele prosjektet er oppdelt i tre faser: Planlegging og overordna design, implementasjon og testing og sluttrapportering. Dette kapittelet gir en overordna beskrivelse av innleveringene som inngår i hver fase i fellesprosjektet. Krav til innleveringer knyttet til hvert enkelt fag finner du på de respektive fagenes hjemmesider eller på ”it’s learning”.

\subsection{Fase 1 – Planlegging og overordnet design}

Det er tre innleveringer tilknyttet til denne fasen. Se Table 1 for informasjon om leveranser og frister. 

\subsubsection{Innlevering: PU1: Prosjektplan}

Prosjektplanen beskriver prosjektets mål og midler, blant annet:

\begin{enumerate}

\item
Hvilke ressurser man har til disposisjon, nemlig penger, personell og utstyr.

\item
Tid og kostnadsestimater for prosjektet

\item
Hvordan arbeidet skal brytes ned / opp i mindre deler

\item
Til hvilke tidsfrister hver del skal være ferdig.

\item
Hvem som skal ha ansvaret for hvilke arbeidsoppgaver.

\end{enumerate}

For mer detaljer om hva planen bør inneholde, samt tips om evt. diagrammer man kan bruke for å få oversikt, se pensum og forelesninger om emnet i TDT4140 Programvareutvikling.

I vårt tilfelle vil nedbrytningen i aktiviteter med tidsfrister langt på vei være gitt av leveranseplanen for prosjektet. Dere bestemmer imidlertid selv når hver enkelt aktivitet skal påbegynnes. Start i god tid, og ha gjerne interne frister underveis mot leveransen, slik at dere kan sjekke innad i gruppen at ting er under kontroll.

Hvem som skal gjøre hva avgjør dere selv, innen visse rammer. Av læringshensyn gjelder følgende krav: Alle gruppemedlemmer skal være involvert i alle fagdeler. Deltagelsen i hver fagdel bør betraktes som en investering i et godt eksamensresultat i faget. Det er dessuten neppe lurt å fordele arbeidet slik at en person skal gjøre hele planleggingen, en annen hele nettdelen, osv. Dette vil gi en ubehagelig ujevn arbeidsbelastning gjennom semesteret. I utgangspunktet bør prosjektplanen legge opp til at man skal jobbe jevnt dag for dag med mindre man har spesielle grunner til noe annet. Siden leveransene er av temmelig ulik størrelse, kan en for grov inndeling føre til at noen sitter igjen med ”svarteper” mens andre ”surfer” i mål med bare enkle arbeidsoppgaver.

Om man har satt opp en prosjektplan med detaljerte personallokering, bør man likevel ikke se på denne som bindende. Det kan være at man finner ut at noe er vanskeligere enn forventet, mens annet er lettere, dvs. at noen har for mye å gjøre og (mens) andre for lite. I så fall bør personallokeringen endres. Dessuten kan man få uforutsett fravær, fordi gruppemedlemmer blir syke eller lignende Dette er det umulig å planlegge på forhånd.

Prosjektplanen må, som et minimum, inneholde følgende kapitler:

\begin{enumerate}

\item
Nedbryting av prosjektet i arbeidspakker – Work Breakdown Structure. Ingen arbeidspakke skal være større enn 16 timeverk.

\item
Kostnadsoverslag – estimerte timeverk – for hver enkelt pakke. Enkeltoverslagene skal kombineres til et totalestimat for prosjektet.

\item
Ganttdiagram, som viser hvordan arbeidspakkene er lagt ut i tid og hvilke avhengigheter som finnes mellom arbeidspakkene. Alle leveransene i Fellesprosjektet skal inn i Ganttdiagrammet. 

\item
Risikoanalysen for prosjektet – hva kan gå galt, hvor sannsynelig er det, hva slags konsekvenser har det og hva kan vi gjøre med det.

\end{enumerate}

\subsubsection{Innlevering: PU2: Systemtestplan}

Systemtesten er en svarteboks test av hele systemet for å se om det tilfredsstiller de spesifiserte krav. Systemtesten kan ikke utføres før alle systemets deler er implementert og integrert, men denne leveransen gjelder planen for testen. Planen skal det være mulig å lage ut fra brukernes krav til systemet, dvs. før design og implementasjon. Alle kravene må testes, både de funksjonelle og de ikke-funksjonelle, og planen må angi hvordan dette skal skje. Derfor må hver enkelt test inneholde:

\begin{enumerate}

\item
Formålet med testen

\item
Hvilken tilstand skal systemet være i

\item
Hvilke data skal gis inn

\item
Hvilken respons man forventet fra systemet hvis det fungerer korrekt

\end{enumerate}

Planen må også si noe om hvordan man skal rapportere eventuelle feil som oppdages under testen.

For detaljer om systemtestplaner og deres innhold, se TDT4140 pensum og foiler fra øvingsforelesninger, eventuelt også mulig støttelitteratur.

\subsubsection{Innlevering: PU3: Overordnet design}

Denne leveransen skal inneholde følgende:

\begin{enumerate}

\item
Use case diagrammer for scenariene i kapitel 2.3.1, 2.3.2 og 2.3.3.

\item
Tekstlige use case for scenariene i kapitel 2.3.2, 2.3.3 og 2.3.4.

\item
Sekvensdiagrammer for scenariene i kapittel 2.3.1 til 2.3.4

\item
Beskrivelse av systemets struktur og de viktigste klassene med tilhørende attributter og metoder.

\end{enumerate}

\subsection{Fase 2 – Implementasjon og testing}

\subsubsection{Innlevering: PU4: Programvare}

Under implementering skal hver gruppe dokumentere unit test og system test prosesser. Ved slutten av fasen skal gruppene levere/demonstrere systemet til und.ass – et system som realiserer alle kravene med eventuelle forbehold som er gitt i kapittel 3.5. Und.ass godkjenner/underkjenner ut i fra gruppas implementerte system og systemet i henhold til gruppas test dokumentasjon.

\subsection{Fase 3 – Sluttrapportering}

\subsubsection{Innlevering: PU5: Dokumentasjon - Sluttrapport, inkludert systemtest og endringsrapport.}

Sluttrapporten skal oppsummere erfaringene fra prosjektet og evaluerer hvorvidt det ble vellykket. Hvis man klarte å levere alt det man skulle innen tidsfristene og endte opp med et programsystem som tilfresstilte de angitte kravene, må dette betraktes som vellykket, særlig hvis man også greide å holde seg inne de angitte ressursrammene – dvs. ikke jobbet mer enn de belastningstimer som var forutsatt. 

For å kunne rapportere tidsbruken, må dere skrive timer på prosjektet. Hvis man har brukt mange flere timer enn planlagt, bør dette stå i sluttrapporten. Angi også hvilke aktiviteter som først og fremst førte til overskridelsen – eller eventuelt hvilke tekniske eller organisatoriske problemer, hvis dette var årsaken. NB! Man vil ikke få noe kritikk eller større fare for å stryke som følge av tidsoverskridelsene. Dette gjelder også underskridelse, så lenge man leverer det man skal. Rapporter tidsbruken så sannferdig som mulig, disse opplysningene er nyttige for oss for å evaluere opplegget og for eventuelt å gjøre forandringer til neste år.

Sluttrapporten vil inkludere systemtestrapporten, som forteller, punkt for punkt, hvordan systemtesten gikk, dvs. hvilke krav som viste seg å være tilfredsstilt og hva som eventuelt ikke fungerte. Man skal også levere en endringsrapport som inneholder opplysninger om rettinger man har gjort som følge av feil funnet i testfasen. For hver endring må man gjøre nye tester for å forsikre seg om at endringen var en suksess. Disse testene må rapporteres på samme måte som andre tester.

Sluttrapporten skal inneholde følgende kapitler:

\begin{enumerate}

\item
Estimert og virkelig tidsforbruk for hver arbeidspakke. For de arbeidspakkene som avviker mer enn 25 \% fra opprinnelig anslag skal det skrives en kort kommentar - ett avsnitt - om de viktigste årsakene til avviket.

Rapporter tidsbruken så sannferdig som mulig. Disse opplysningene er nyttige for oss når vi skal evaluere opplegget og eventuelt gjøre forandringer neste år

\item
Systemtestrapporten, som skal inneholde:

\begin{enumerate}

\item
Resultatene fra systemtesten, dvs. hvilke krav som ble tilfredsstilt og hva som eventuelt ikke fungerte. Bruk samme nummerering som i kravdelen av dette kompendiet.

\item
Endringsrapport, som viser hvilke rettinger man har gjort som følge av feil funnet under systemtesten. For hver endring må man gjøre nye tester for å test at endringen var vellykket. Disse testene må rapporteres på samme måte som andre tester i systemtesten.

\end{enumerate}

\item
Prosjekterfaringer. Som et minimum må dette kapitlet inneholde et avsnitt om hver av de fire viktigste tingene dere vil legge vekt på å 

\begin{enumerate}

\item
Unngå i senere utviklingsprosjekter.

\item
Gjenta i senere utviklingsprosjekter.

\item
Forbedre dere for senere utviklingsprosjekter.

\end{enumerate}

\end{enumerate}
